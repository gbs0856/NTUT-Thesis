%
% Bibliography, index, etc., in TOC
% http://www.tex.ac.uk/cgi-bin/texfaq2html?label=tocbibind
%
% Appendixes
% http://www.tex.ac.uk/cgi-bin/texfaq2html?label=appendix
%
% Watermark
% http://www.tex.ac.uk/cgi-bin/texfaq2html?label=watermark
%
% 頁碼的編排
% http://www.econ.ntu.edu.tw:8080/tmwu/viewtopic.php?t=433&highlight=%A5%D8%BF%FD+%BDs%BDX
%
% 使用chapter的方法
% \titleformat{\chapter}[display]{\vspace*{-5mm} \centering \bfseries \fontsize{20pt}{18pt} \selectfont}{Chapter \thechapter}{0.2cm}{}
% \chapter{Introduction}
%
% 使用section的方法
% \titleformat{\section}[hang]{\fontsize{18pt}{0pt} \selectfont \bf}{\thesection}{0.2cm}{}
% \section{Broadcast for Index Trees}
%
% 使用subsection的方法
% \titleformat{\subsection}[hang]{\fontsize{16pt}{0pt} \selectfont \bf}{\thesubsection}{0.2cm}{}
% \subsection{subsection}
%
% 使用subsubsection的方法
% \titleformat{\subsubsection}[hang]{\fontsize{14pt}{0pt} \selectfont \bf}{\thesubsubsection}{0.2cm}{}
% \subsubsection{subsubsection}
%

% 字型大小12點, A4的紙張, 內文格式採用report
\documentclass[12pt,a4paper]{report}
%\documentclass[12pt,a4paper]{article}
%

% 引用套件
% 使用中文套件xeCJK時,無法使用字型套件times
%\usepackage{times}
% 中文套件
\usepackage{xeCJK}
% 設定中文為系統上的字型,而英文不去更動,使用原TeX字型
% for windows
%\setCJKmainfont[AutoFakeBold=4,AutoFakeSlant]{標楷體}
% for overleaf
\setCJKmainfont[AutoFakeBold=4,AutoFakeSlant]{TW-Kai}
% 這兩行一定要加,中文才能自動換行 
\XeTeXlinebreaklocale "zh"
\XeTeXlinebreakskip = 0pt plus 1pt
% 設定非中文字型
\setmainfont{Times New Roman}

%\usepackage{fancyhdr}
\usepackage{amsthm}
\usepackage[fleqn]{amsmath}
\usepackage{amssymb}
%bm是可以讓數學符號變粗體 使用方式 \bm{符號}
%\usepackage{bm}
\usepackage{graphicx}
\usepackage{subfigure}
\usepackage{setspace}
\usepackage{float}
\usepackage{longtable}
\usepackage{rotating}
\usepackage[algo2e,linesnumbered,boxruled]{algorithm2e}
\usepackage{pdfpages}
\usepackage[nottoc,notlot,notlof]{tocbibind}

% 圖的編號格式設定(英文版)
\usepackage[labelsep=period]{caption}
% 圖的編號格式設定(中文版)
%\usepackage[labelsep=quad]{caption}
%\renewcommand{\figurename}{圖}
%\renewcommand{\tablename}{表}


% 版面與邊界設定
\usepackage[nohead,top=2.5cm,left=2.5cm,right=2.5cm,bottom=2.75cm,footskip=1cm]{geometry}

%只在twoside(雙面列印)時有作用
%讓章節結束在偶數頁,若為奇數頁則自動補空白頁
\newcommand{\chapterfinished}{\clearpage
	\ifodd\count0 \else \thispagestyle{empty}\mbox{}\clearpage \fi}

% 紙張背景(學校LOGO浮水印)
\usepackage{background}
%取消浮水印
\newcommand\DeactivateBG{
	\backgroundsetup{contents={}}
}
%開始浮水印
\newcommand\ActivateBG{
	\backgroundsetup{scale=1, angle=0, opacity=0.13, color =black, contents={\includegraphics[scale=1]{background.png}}}
}

%在封面、書名頁後插入無浮水印之空白頁(雙面列印才需使用)
\newcommand\newblankpage{\clearpage\DeactivateBG\thispagestyle{empty}\mbox{}\clearpage\ActivateBG
}

% tocloft 建立目錄之套件
\usepackage[subfigure]{tocloft} 
%\usepackage{titletoc} %此套件雖新,但無法自定義目錄清單項目開頭標籤,如Table 1.X或Figure 2.X
% titlesec 自訂章節樣式之套件
\usepackage{titlesec}

%中文段落縮排
%\usepackage{indentfirst} \setlength{\parindent}{24pt}

%數學公式 編號格式 : 章節數.編號
\numberwithin{equation}{chapter}


% 將section編號設定為三層
\setcounter{secnumdepth}{3}
%

% 設定目錄顯示三層section
\setcounter{tocdepth}{3}
%

% 將Content (Tables, Figures)改成Table of Contents(List of Tables, List of Figures)
%\renewcommand{\contentsname}{\protect \centering TABLE OF CONTENTS}
%\renewcommand{\listtablename}{\protect \centering LIST OF TABLES}
%\renewcommand{\listfigurename}{\protect \centering LIST OF FIGURES}
\renewcommand*\cfttoctitlefont{ \vspace{-30pt}\hfil\bfseries\fontsize{20pt}{18pt} \selectfont}
\renewcommand*\cftloftitlefont{ \vspace{-30pt}\hfil\bfseries\fontsize{20pt}{18pt} \selectfont}
\renewcommand*\cftlottitlefont{ \vspace{-30pt}\hfil\bfseries\fontsize{20pt}{18pt} \selectfont}
\renewcommand*\contentsname{TABLE OF CONTENTS}
\renewcommand*\listfigurename{LIST OF FIGURES}
\renewcommand*\listtablename{LIST OF TABLES}
% 將Bibliography改為BIBLIOGRAPHY
\renewcommand*\bibname{BIBLIOGRAPHY}

% 使用tocloft套件取消套件自建各種目錄的標題
\makeatletter
\renewcommand{\@cftmaketoctitle}{}
\renewcommand{\@cftmakelottitle}{}
\renewcommand{\@cftmakeloftitle}{}
\makeatother

%自訂目錄清單項目開頭標籤與標籤大小
\renewcommand{\cfttabpresnum}{Table } % Adds the word &quot;Figure&quot; in front of the number.
\renewcommand{\cfttabnumwidth}{5em} % Adds some white space.
\renewcommand{\cftfigpresnum}{Figure } % Adds the word &quot;Figure&quot; in front of the number.
\renewcommand{\cftfignumwidth}{5em} % Adds some white space.


% 自訂英文chapter命令,並定義在目錄中顯示Chapter的格式
% 使用方式 \echapter{章節數}{名稱} ex:\echapter{1}{introduction}
\newcommand{\echapter}[2]{
	\setcounter{chapter}{#1}
	\setcounter{section}{0}
	\setcounter{table}{0}
	\setcounter{figure}{0}
	\setcounter{equation}{0}
	\chapter*{Chapter #1\\\MakeUppercase{#2}}
	\vspace{-1em}
	\addcontentsline{toc}{chapter}{Chapter #1\quad\MakeUppercase{#2}}
}
% 自訂中文chapter命令,並定義在目錄中顯示Chapter的格式
% 使用方式 \cchapter{章節數(數字)}{章節數(中文)}{名稱} ex:\cchapter{1}{一}{背景}
\newcommand{\cchapter}[3]{
	\setcounter{chapter}{#1}
	\setcounter{section}{0}
	\setcounter{table}{0}
	\setcounter{figure}{0}
	\setcounter{equation}{0}
	\chapter*{第#2章\quad #3}
	\vspace{-1em}
	\addcontentsline{toc}{chapter}{第#2章\quad #3}
}

%
% 行距調成1.5倍行高
\linespread{1.5}
% 行距調成1.9倍行高 = Word的1.5倍行高
%\linespread{1.9}
%

% 定義的指令
\newtheorem{thm}{Theorem}
\newtheorem{lem}[thm]{Lemma}
\newtheorem{defi}{Definition}
\newtheorem{prop}{Property}
%
%更換證明結束符號為黑色方塊
\renewcommand\qedsymbol{$\blacksquare$}

% 演算法設定
\SetAlCapSkip{1em}
\SetAlCapHSkip{1em}

%hyperref套件是讓所有交互參照和目錄加上連結,此套件需最後載入否則一堆奇怪問題
%目前會無法辨認自訂的章節命令,如本檔中自訂的\echapter和\cchapter
%因此Log會出現Error訊息,但能成功編譯(不包含bibtex)
%請先註解掉這段,編譯過3次並有正常cite編號後再使用
%hidelinks=隱藏連結框+字不變色
%pdfauthor={作者姓名}
%pdftitle={論文名稱}
%pdfkeywords={關鍵字清單}
\usepackage[breaklinks=true]{hyperref}
\hypersetup{hidelinks,
pdfstartview=Fit, pdfauthor={姓名}, pdftitle={論文名稱}, pdfkeywords={keyword1, keyword2, keyword3, keyword4, and keyword5}
}


\begin{document}
% 取消使用浮水印
\DeactivateBG
% 外封面
\noindent\hspace*{.1cm}\includegraphics[width=0.207\textwidth]{Logo.png}\hspace{1.15cm}\includegraphics[width=0.53\textwidth]{school_name.png}
\begin{center}
\begin{spacing}{1}
	\fontsize{24pt}{36pt}\selectfont 
	\bf 資訊工程系博士班\\
	博士學位論文\\
	\vspace{95pt}					%距離請自己調整
	\fontsize{24pt}{32pt}\selectfont 
	\bf 論文中文標題第一行\\論文中文標題第二行\\	
	\fontsize{22pt}{32pt}\selectfont 
	English Title\\
	\vspace{140pt}					%距離請自己調整
	\fontsize{18pt}{18pt}\selectfont
	\bf 研究生:XXX\\
	\vspace{72pt}					%距離請自己調整
	\bf 指導教授:姚立德 博士\\	
	\vspace{54pt}					%距離請自己調整
	\bf 中華民國一百零六年一月
\end{spacing}
\end{center}
% 外封面無頁碼
\thispagestyle{empty}
% 結束外封面
% 新增無頁碼和浮水印空白頁(雙面列印用)
\newblankpage

%使用浮水印
\ActivateBG
% 內封面(書名頁)
\begin{center}
\begin{spacing}{1}
	\vspace*{10pt}					%距離請自己調整
	\fontsize{24pt}{36pt}\selectfont 
	\bf 國立臺北科技大學\\
	資訊工程系博士班\\
	博士學位論文\\
	\vspace{95pt}					%距離請自己調整
	\fontsize{24pt}{32pt}\selectfont 
	\bf 論文中文標題第一行\\論文中文標題第二行\\	
	\fontsize{22pt}{32pt}\selectfont 
	English Title\\
	\vspace{140pt}					%距離請自己調整
	\fontsize{18pt}{18pt}\selectfont
	\bf 研究生:XXX\\
	\vspace{72pt}					%距離請自己調整
	\bf 指導教授:姚立德 博士\\	
	\vspace{54pt}					%距離請自己調整
	\bf 中華民國一百零六年一月
\end{spacing}
\end{center}
% 內封面無頁碼
\thispagestyle{empty}
% 結束內封面
% 新增無頁碼和浮水印空白頁(雙面列印用)
\newblankpage


% 取消使用浮水印
\DeactivateBG
%插入審定書PDF檔,等名預設為Sign
\includepdf[pages={1}]{Sign}
\newblankpage

% 從這裡開始頁碼用羅馬數字(重新編號)
\pagenumbering{roman}
%

% 摘要、目錄、表目錄、圖目錄的標題設定
\titleformat{\chapter}[hang]{\protect \vspace{-36pt} \centering \bfseries \fontsize{20pt}{18pt} \selectfont}{}{}{}


\begin{spacing}{1.5}
% 中文摘要標題
\chapter*{摘 要}
% 將中文摘要標題加入目錄
\addcontentsline{toc}{chapter}{摘要}
% 引入摘要內容
論文名稱:XXXXXXXXXXXXXXXXXXXXXXXXXXXXXXXXXXXXXX

\noindent 頁數:一百XX頁

\noindent 校所別:國立臺北科技大學 XX工程 研究所

\noindent 畢業時間:一百XX學年度 第X學期

\noindent 學位:博士

\noindent 研究生:XXX

\noindent 指導教授:XXX 博士\\

\noindent 關鍵詞:XXX網路、XXX網路、XXX網路\\

近年來,XXXXXXXXXXXXXXXXXXXXXXXXXXXX。
% 結束摘要
% 單面印刷用\clearpage,雙面印刷用\chapterfinished
%\clearpage
\chapterfinished


% 英文摘要標題
\chapter*{ABSTRACT}
% 將英文摘要標題加入目錄
\addcontentsline{toc}{chapter}{ABSTRACT}
% 引入摘要內容
Title: XXXXXXXXXXXXXXXXXXXXXXXXXXXXXXXXXXXXXXXXXXXXXX

\noindent Pages: 1XX

\noindent School: National Taipei University of Technology

\noindent Department: Computer Science and Information Engineering

\noindent Time: January, 2017

\noindent Degree: Ph.D

\noindent Researcher: XXXXXX

\noindent Advisor: XXXXX, Ph.D\\

\noindent Keywords: XXX Networks, XXX Networks, XXX Networks.\\

Recent advancements in mobile device computing power, wireless communications, and networking technology have made it easier to gather useful data through mobile device sensors. The integration of mobile sensors with GPS has led to the emergence of a variety of intriguing applications and research topics. This dissertation discusses XXXXXXXXXXXXXXXX.

% 結束摘要
% 單面印刷用\clearpage,雙面印刷用\chapterfinished
%\clearpage
\chapterfinished


% 英文誌謝標題
\chapter*{ACKNOWLEDGMENT}
% 將誌謝標題加入目錄
\addcontentsline{toc}{chapter}{ACKNOWLEDGMENT}
% 引入誌謝內容
I am very grateful to my advisor Prof. Chuan-Ming Liu for providing me with continuous support and guidance throughout my PhD program which has been immensely helpful to me. My parents and sister provided the emotional support needed to take up this challenge and complete it successfully. Akbar and Nick - without your support this journey would not have been possible.

I want to thank Prof. Ya-Hui Chang, Prof. Jen-Yeu Chen, Prof. Kuan-Chou Lai, Prof. Prof. Li-Chun Wang, and Prof. Chang-Wu Yu for serving on my committee to provide their guidance on improving this dissertation. I am also indebted to a number of other professors from whom I gained immensely valuable knowledge over the course of the past five years.

I am thankful to the Department of Computer Science and Information Engineering, the College of Electrical Engineering and Computer Science, the Graduate School of National Taipei University of Technology, and the Ministry of Science of Technology for providing the needed financial support. The research work carried out in this dissertation would not have been possible without the financial support from these institutions and organizations. The opportunities provided by the  Department of Computer Science and Information Engineering to work as a teaching assistance not only supported me financially but also provided a valuable experience that would benefit me in the future.

I am grateful and fortunate to have a number of close, supportive, and entertaining friends who made this journey enjoyable and memorable. Doris, Henry, Ian, Paul, Sky, Stanley, and Winny - thank you for giving me the most memorable and awesome years of my life.
% 結束誌謝
% 單面印刷用\clearpage,雙面印刷用\chapterfinished
%\clearpage
\chapterfinished

% 目錄
\chapter*{TABLE OF CONTENTS}
\addcontentsline{toc}{chapter}{TABLE OF CONTENTS}
\tableofcontents
% 結束目錄
% 單面印刷用\clearpage,雙面印刷用\chapterfinished
%\clearpage
\chapterfinished

% 表目錄
\chapter*{LIST OF TABLES}
\addcontentsline{toc}{chapter}{LIST OF TABLES}
\listoftables
% 單面印刷用\clearpage,雙面印刷用\chapterfinished
%\clearpage
\chapterfinished

% 圖目錄
\chapter*{LIST OF FIGURES}
\addcontentsline{toc}{chapter}{LIST OF FIGURES}
\listoffigures
% 單面印刷用\clearpage,雙面印刷用\chapterfinished
%\clearpage
\chapterfinished

\end{spacing}

% 從這裡開始頁碼用阿拉伯數字(重新編號)
\pagenumbering{arabic}




% 正文Chapter標題格式設定
\titleformat{\chapter}[display]{\protect \vspace*{-49pt} \centering \bfseries \fontsize{20pt}{30pt} \selectfont}{Chapter \thechapter}{0.2cm}{}
% 正文Section標題格式設定
\titleformat{\section}[hang]{\fontsize{18pt}{27pt} \selectfont \bf}{\thesection}{0.2cm}{}
% 正文subsection標題格式設定
\titleformat{\subsection}[hang]{\fontsize{16pt}{24pt} \selectfont \bf}{\thesubsection}{0.2cm}{}
% 正文subsubsection標題格式設定
\titleformat{\subsubsection}[hang]{\fontsize{14pt}{21pt} \selectfont \bf}{\thesubsubsection}{0.2cm}{}





% 正文開始依章節順序導入
% 章: Introduction
\cchapter{1}{一}{背景} \label{ch:1}  %中文標題請用\cchapter
%\echapter{1}{Introduction} \label{ch:1}
Thanks to advances in modern technologies such as wireless communications, networking, information management, and positioning systems, wireless sensors have become common and are now integrated into mobile devices. For instance, the Samsung Galaxy S4 was the first major smartphone to include a barometer, thermometer, and hygrometer (to measure humidity), allowing people to use their mobile devices to easily gather environmental information. This shift has sparked interest in emerging mobile environments like Mobile Ad-hoc Networks (MANETs), Mobile Wireless Sensor Networks (MWSNs), Mobile Peer-to-Peer Networks (MP2P), and Data Broadcasting environments.

The widespread adoption of the Global Positioning System (GPS) and the miniaturization of mobile devices equipped with wireless network modules have made location-based services (LBSs) more affordable. The combination of mobile sensors and GPS has led to the emergence of intriguing applications and research opportunities. In wireless mobile sensor networks, each mobile device acts as a moving object, using GPS and sensors to provide environmental data such as location, temperature, and humidity to information systems. These systems can collect user data and send user location through mobile networks, enabling immediate access to location-specific information. Manufacturers now offer a range of telecommunication services, enhancing convenience for users. For example, if a user queries the temperature at a particular location, nearby mobile devices will return measured temperature data. Similarly, a mobile sink might move to gather temperature data in a temperature monitoring system. In these scenarios, the query can move.

\section{Challenges 第一階層子標題}
Before we present the detailed proposed approaches for processing C$k$NN and CRSQ queries, in this section we give the overviews of considered emerging mobile environments.

\subsection{Mobile Wireless Sensor Networks}
The Mobile Wireless Sensor Network (MWSN) is an emerging technology with significant applications. A MWSN can simply YYYYYYYYYYYYYYYYYYYYYYY.

\subsection{Mobile Peer-to-Peer Networks}
但在廣播環境中進行道路網路的適地性查詢也有難處,伺服端無法事先知道查詢點的位置以及客戶端在下了查詢後亦無法隨意取得自己需要的區塊資料。因此,需要先預先處理需要的資料,將這些資料放進在廣播資料裡,輔助客戶端能篩選出自己需要的區塊,但這些資料也不能佔據太多的廣播空間,不然客戶端會需要很長的時間才能計算出答案,這些預先處理的資料我們稱為索引結構。本篇論文主要探討在廣播環境中的反向最近鄰居搜尋,以下是本論文提出的方法。在廣播伺服端方面,伺服端會先將道路網路利用網格切割法分割成數個區塊,接著預先計算所有邊界點對於相鄰區塊裡的最近景點距離以及每個區塊內部的最大距離後加入索引結構;在客戶端方面,客戶端以相應的方式接收索引,並利用索引內的資料計算出整個查詢的距離上限(Upper Bound),與每個區塊與查詢點各自的距離下限(Upper Bound),利用這些距離上限與下限,可以篩選出客戶端可能需要的區塊,當客戶端接收完所有可能區塊後,便可算出查詢的答案。

\section{Motivation}
In this section, we present some concepts and definitions that are generally used about the spatial data queries on moving data objects. 

\subsection{\textit{k} Nearest Neighbors Search}
The nearest neighbors (NN) search is a fundamental problem in computer science that is crucial for applications such as geographic information systems (GIS) and pattern recognition. With the increasing popularity of information services related to NN search, there has been significant research on this topic, resulting in many variations of NN search queries, such as group nearest neighbors (GNN), $k$ nearest neighbors, range nearest neighbors, line nearest neighbors, and reverse nearest neighbors. Of these, the $k$ nearest neighbors ($k$NN) search is the most widely used. A mobile client might query "where is the nearest gas station?" or "please provide the five closest bookstores." Generally, the goal of a $k$NN search at a query node $q$ is to identify the $k$ closest objects to $q$ within a given data set $D$. The function $dist(a, b)$ denotes the distance between two objects, $a$ and $b$. In Euclidean space, the distance is defined as Definition~\ref{d1}.
%
\begin{defi}
	\label{d1}(Euclidean distance)\\
	Given two data objects (or points) $a$ and $b$ in a plane, we denote the distance between two objects $a=(a_x,a_y)$ and $b=(b_x,b_y)$ as
	$dist(a,b)=\sqrt{(a_x-b_y)^2+(a_y-b_y)^2}$.
\end{defi}

\clearpage

\section{Research Objectives}
Existing works on spatial query processing can be classified into three categories based on how data is accessed or managed: (1) XXX approach, (2) XXX approach, and (3) XXXX approach. The pull-based approach follows a traditional client-server model, with a central server handling spatial queries from user devices and storing all data. When a client requests information, it sends a query to the server. Since the amount of spatial data that needs processing can be substantial, the data is often stored in external storage, such as disks. Additionally, communication bandwidth between clients and servers is asymmetric, with limited uplink bandwidth compared to downlink. This disparity in wireless communication can lead to bottlenecks when the number of queries increases if the pull-based approach is used for data access.




\section{Dissertation Outline}
This dissertation focuses on two specific research issues in emerging mobile environments: spatial query processing and data synchronization. In the next chapter, XXXX XXXXXXXXX XXXXXX XXXXXXXXXX. First, techniques for XXXXX are introduced. Then, the problems of data synchronization in XXXX environments and XXX issues in XXXXX information systems are discussed. Finally, the dissertation's main research contributions are outlined.

Chapter 3 addresses XXXXXXXXXXXX

Chapter 4 explores XXXXXXXXXXXXXXX

The final chapter concludes the dissertation and outlines potential directions for future research topics.
% 單面印刷用\clearpage,雙面印刷用\chapterfinished
%\clearpage
\chapterfinished

% 章: State of the Art and Contributions
\echapter{2}{State of the art and contributions} \label{ch:2}
In this chapter, we discuss the related works about the specific research problems addressed in this dissertation. 

\section{Techniques for XXX}
In spatial-temporal data applications, datasets comprise data objects, and both the data and queries can evolve over time. To efficiently handle large volumes of spatial data and queries, existing works~\cite{Chon:2003:RKN:942343.942352} have employed decomposition methods like grids and Voronoi diagrams..



 
\label{ch:query-processing}

\section{Techniques for XXX2}
In the unstructured MP2P systems, the general solution for increasing the data accessibility is to replicate data items in the distributed cache on different nodes. It is difficult and complicated to maintain data items and keep the consistency of replicas in the MP2P systems because of the unpredictable mobility of the nodes. However, the consistency of replicas should be guaranteed; otherwise, the users may obtain the dirty or old data. Thus, the consistency of data items and replicas has become an important issue on the data management in unstructured MP2P systems and many approaches have been proposed to settle the data consistency problem in the unstructured MP2P~\cite{springerlink:10.1007/s11277-006-9238-z}.


\section{Research Contributions}


Specifically, the research work presented in this dissertation has following contributions.
\begin{itemize}
	\item \textbf{Field 1}
	\begin{itemize}
		\item XXXXXXXXXXXXXXXXXX
		\item XXXXXXXXXXXXXXXXX
		\item XXXXXXXXXXXXXXXXXXX 
		\item XXXXXXXXXXXXXXXXXX
	\end{itemize}
	\item \textbf{Field 2}
	\begin{itemize}
		\item WXXXXXXXXXXXXXXX
		\item XXXXXXXXXXXXXXX
		\item XXXXXXXXXXXXXXXXXXXXXX
		\item XXXXXXXXXXXXXXXXXXXX
	\end{itemize}
\end{itemize}

%The above contributions are detail in chapter 3 and chapter 4.

% 單面印刷用\clearpage,雙面印刷用\chapterfinished
%\clearpage
\chapterfinished


% 章: Query Processing on Mobile Wireless Sensor Networks
\echapter{3}{System Model and Problem Formulation} \label{ch:3}
In this chapter, we propose new approaches which XXXXXXXXXXXXXXXXXXX. 

\section{System Model}
Before introducing the proposed algorithm, we first give some background and terminologies. And the metrics generally used to evaluate the query process will also be introduced.

\subsection{Voronoi Cell}


XXXXXXXXXXXXXXXXXXXXX

\section{Problem Formulation}
\input{3.2-ProblemFormulation}
% 單面印刷用\clearpage,雙面印刷用\chapterfinished
%\clearpage
\chapterfinished


% 章: The Proposed Method
\echapter{4}{The Proposed Method} \label{ch:4}

\section{Proposed Method}
We introduce a distributed algorithm, DCNN, for CNN search in mobile and distributed environments. This approach, which utilizes a better safe-time, helps reduce update costs, lessen server load, shorten response times, and enhance result accuracy. DCNN consists of three phases: initialization, query processing, and information update.


\section{Complexity Analysis}
4.2 中文測試
% 單面印刷用\clearpage,雙面印刷用\chapterfinished
%\clearpage
\chapterfinished


% 章: The Proposed Method
\echapter{5}{Simulation Results} \label{ch:5}

\section{Simulation Settings}
In this section, all simulations were run on a Windows 7 system with an Intel i5-4460 3.20GHz CPU and 8GB of memory. For all simulation experiments, the final results represent the average of 100 queries. The mobility model used is Random Waypoint (RWP), while the underlying routing protocol is AODV.

\subsection{Performance Metrics}
In this subsection, we introduce all the performance metrics t

\emph{1) Update Cost (Frequency)}\\
\indent



\emph{2) Number of Accessed Nodes}\\
\indent


\emph{3) Number of Messages}\\
\indent


%\subsection{Response Time}
\emph{4) Response Time}\\
\indent


%\subsection{Accuracy}
\emph{5) Accuracy}\\
\indent


\emph{5) Precision}\\
\indent


\subsection{Performance of XXXXX}
XXXXXXXXXXXXXXXXXX

\subsubsection{subsubsection}
test test 測試 測試 測試 測試 測試 測試 測試 測試 測試 測試 測試



\clearpage


\section{Discussion}
The performance of our proposed approaches are validated by the extensive simulation experiments. 
% 單面印刷用\clearpage,雙面印刷用\chapterfinished
%\clearpage
\chapterfinished



% 章: Research Plan
\echapter{6}{Conclusion and Future Directions} \label{ch:conclusion}
This dissertation introduces efficient distributed query processing and data consistency management approaches for mobile networks. These methods establish fundamental principles for handling queries and maintaining shared data in distributed environments. However, with the rapid advancement of wireless technologies and information sciences, computing environments are evolving quickly. Many applications are significantly affected by Big Data generated from human behavior, giving rise to new networks and applications such as Delay-Tolerant Networks (DTNs), Mobile Social Networks (MSNs), and Mobile Edge Computing (MEC). In light of these developments, the dissertation could potentially be extended to address these emerging environments in the future.
% 單面印刷用\clearpage,雙面印刷用\chapterfinished
%\clearpage
\chapterfinished


{
\setstretch{1.2}
% 參考文獻格式設定
%\bibliographystyle{plain}
\bibliographystyle{abbrv}
% 參考文獻標題格式設定
\titleformat{\chapter}[hang]{\protect \vspace{-30pt} \centering \bfseries \fontsize{20pt}{18pt} \selectfont}{}{}{}
%\titlecontents{chapter}[6mm]{\vspace{18pt}}{\bfseries \contentslabel{6mm}}{\hspace*{-7mm} \bfseries \contentslabel{8mm}}{\titlerule*[10pt]{} \bfseries \contentspage}

%\chapter*{BIBLIOGRAPHY}
% 將參考文獻加入目錄
%\addcontentsline{toc}{chapter}{BIBLIOGRAPHY}
%引入文獻檔,使用方式為 \bibliography{檔名}
\bibliography{reference}
}
\chapterfinished

% 將符號表的標題格式設定成跟目錄相同
\titleformat{\chapter}[hang]{\protect \vspace{-36pt} \centering \bfseries \fontsize{20pt}{18pt} \selectfont}{}{}{}
% 符號說明清單
\chapter*{LIST OF SYMBOLS}
\label{symbols}
% 將符號說明清單加入目錄
\addcontentsline{toc}{chapter}{LIST OF SYMBOLS}
% 引入符號說明清單內容
{
\renewcommand*{\arraystretch}{1.2}
\begin{longtable}{p{.20\textwidth} p{.80\textwidth}}
		\textbf{Symbol} & \textbf{Description} \\
		$n_i$, $o_i$ & A mobile node (data object)\\
		$C_{n_i}$ & The network connectivity of node $n_i$ \\
		$c_{\varDelta t}$ & The number of successfully updated nodes in a time period $\varDelta t$ \\
		$S$ & The set of all the shared data objects\\
		
		$v$ & Node speed \\
		$W$ & The bandwidth of a node \\
\end{longtable}
}
% 結束符號說明清單

\end{document}
