Existing works on spatial query processing can be classified into three categories based on how data is accessed or managed: (1) XXX approach, (2) XXX approach, and (3) XXXX approach. The pull-based approach follows a traditional client-server model, with a central server handling spatial queries from user devices and storing all data. When a client requests information, it sends a query to the server. Since the amount of spatial data that needs processing can be substantial, the data is often stored in external storage, such as disks. Additionally, communication bandwidth between clients and servers is asymmetric, with limited uplink bandwidth compared to downlink. This disparity in wireless communication can lead to bottlenecks when the number of queries increases if the pull-based approach is used for data access.


