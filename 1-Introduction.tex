Thanks to advances in modern technologies such as wireless communications, networking, information management, and positioning systems, wireless sensors have become common and are now integrated into mobile devices. For instance, the Samsung Galaxy S4 was the first major smartphone to include a barometer, thermometer, and hygrometer (to measure humidity), allowing people to use their mobile devices to easily gather environmental information. This shift has sparked interest in emerging mobile environments like Mobile Ad-hoc Networks (MANETs), Mobile Wireless Sensor Networks (MWSNs), Mobile Peer-to-Peer Networks (MP2P), and Data Broadcasting environments.

The widespread adoption of the Global Positioning System (GPS) and the miniaturization of mobile devices equipped with wireless network modules have made location-based services (LBSs) more affordable. The combination of mobile sensors and GPS has led to the emergence of intriguing applications and research opportunities. In wireless mobile sensor networks, each mobile device acts as a moving object, using GPS and sensors to provide environmental data such as location, temperature, and humidity to information systems. These systems can collect user data and send user location through mobile networks, enabling immediate access to location-specific information. Manufacturers now offer a range of telecommunication services, enhancing convenience for users. For example, if a user queries the temperature at a particular location, nearby mobile devices will return measured temperature data. Similarly, a mobile sink might move to gather temperature data in a temperature monitoring system. In these scenarios, the query can move.